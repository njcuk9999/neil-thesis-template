% Use this format to 'store' the glossary entries for the rest of the 
% document. The 'include{glossary}' should be located before the first 
% reference to the glossary. This gives me a way of keeping all glossary 
% terms in one file and just reference the 'labels' in the rest of the 
% document like this : 

%	eg. \useglosentry{btw}
%

%\storeglosentry{}{name={},description={}}
%\storeglosentry{}{name={},description={}}
%\storeglosentry{}{name={},description={}}
%\storeglosentry{}{name={},description={}}
%\storeglosentry{}{name={},description={}}
%\storeglosentry{}{name={key},description={key is the name to use}}

%\useglosentry{key}


% If using the newcommands.tex can use \acro{key}
% The difference between \useglosentry and \acro is that \acro will add the instance to the index as well (to avoid having to use two commands to do this).

%----------------------------------------------------------------------------------------------------------------------% Start of Glossary
%----------------------------------------------------------------------------------------------------------------------

\storeglosentry{2MASS}{name={2MASS},description={The Two ({\bf 2}) {\bf M}icron {\bf A}ll {\bf S}ky {\bf S}urvey (implied point source catalogue) is an all-sky near-infrared ($J$, $H$, $K_S$) catalogue of 470 992 970 objects from \protect\citet{Skrutskie2006}}}

\storeglosentry{WISE}{name={WISE},description={The {\bf W}ide-Field {\bf I}nfrared {\bf S}urvey {\bf E}xplorer is a space based near-to-mid infrared telescope (3.4, 4.6, 12 and 22 \micron), the all-sky source catalogue contains 563 921 584 objects \citep{Wright2010}.}}

\storeglosentry{NIR}{name={NIR},description={The {\bf N}ear {\bf I}nfrared {\bf R}ed is the wavelength region from $\sim$0.7/0.8 to 2.5\micron in the electromagnetic spectrum.}}

\storeglosentry{MIR}{name={MIR},description={The {\bf M}id {\bf I}nfrared {\bf R}ed is the wavelength region from $\sim$2\micron to $sim$ 20 \micron in the electromagnetic spectrum.}}

\storeglosentry{UV}{name={UV},description={The {\bf U}ltra{\bf V}iolet is the wavelength region from $\sim$ 400 nm to 10 nm in the electromagnetic spectrum.}}

\storeglosentry{FWHM}{name={FWHM},description={The {\bf F}ull {\bf W}idth {\bf H}alf {\bf M}aximum is the width of the independent variable at which the independent variable is half of its maximum value.}}

\storeglosentry{SED}{name={SED},description={The {\bf S}pectral {\bf E}nergy {\bf D}istribution is a plot of the brightness or flux density as a function of frequency or wavelength.}}

\storeglosentry{PSF}{name={PSF},description={The {\bf P}oint {\bf S}pread {\bf F}unction describes the response of an imaging system to a point source.}}

\storeglosentry{SNR}{name={SNR},description={The {\bf S}ignal to {\bf N}oise {\bf R}atio, or S/N is a measure of quality of some data compared to its uncertainties.}}

\storeglosentry{MCMC}{name={MCMC},description={The {\bf M}arkov {\bf C}hain {\bf M}onte {\bf C}arlos methods are a class of algorithms for sampling from a probability distribution.}}

\storeglosentry{UCD}{name={UCD},description={{\bf U}ltra{\bf C}ool {\bf D}warf is the collective name for any object of mass less than a spectral type of $\sim$M7, and includes most brown dwarfs and giant planets.}}