% % % % \input{newcommands} must be added to root tex file for these to work
%
% This is basically where I define my new commands
%
% format is normally as follows:
%
%	\newcommand{\command}{stuff here will be executed on use of command}
%
%	then in text use \command to execute "stuff"
%
% ------------------------------------------------------------------------------
% Physics
% ------------------------------------------------------------------------------
\newcommand{\arcsec}{\,\mbox{arcsec}\,}
\newcommand{\Teff}{\mbox{$T_{eff}$}\,}
\newcommand{\logg}{\mbox{$log$\,$g$}\,}
\newcommand{\Msun}{\mbox{$M_{\odot}$}\,}
\newcommand{\Mjup}{\mbox{$M_{Jup}$}\,}
\newcommand{\Rsun}{\mbox{$R_{\odot}$}\,}
\newcommand{\Lsun}{\mbox{$L_{\odot}$}\,}
\newcommand{\micron}{\mbox{${\mu}$m}\,}
%\newcommand{\arcsec}{\mbox{$^{\prime\prime}$}\,}
\newcommand{\Rv}{\mbox{$R_{\text v}$}\,}
\newcommand{\Av}{\mbox{$A_{\text v}$}\,}
\newcommand{\Alam}{\mbox{$A_{\lambda}$}\,}
\newcommand{\EBV}{\mbox{$E(B - V)$}\,}
\newcommand{\halpha}{\mbox{H$\alpha$}\,}
\newcommand{\nai}{\mbox{Na{\sc i}}\,}
\newcommand{\MJ}{\mbox{$M_{J}$}\,}
\newcommand{\MV}{\mbox{$M_{V}$}\,}
\newcommand{\HJ}{\mbox{$H_{J}$}\,}
\newcommand{\HV}{\mbox{$H_{V}$}\,}
\newcommand{\degs}{\mbox{$^{\circ}$}\,}
\newcommand{\Mpc}{\mbox{$Mpc$}\,}
\newcommand{\Myr}{\mbox{$Myr$}\,}
\newcommand{\Gyr}{\mbox{$Gyr$}\,}
%--------------------------------------------------------------------------------
% Maths
%--------------------------------------------------------------------------------
\newcommand{\simm}{$\sim$}
\newcommand{\asin}{sin$^{-1}$}
\newcommand{\la}{\lesssim}
\newcommand{\ga}{\gtrsim}
\newcommand{\chis}{$\chi$-squared}
\newcommand{\chiS}{$\chi$-Squared}
\newcommand{\tx}{\mbox{$\times10$}}

%--------------------------------------------------------------------------------
% Language
%--------------------------------------------------------------------------------
\newcommand{\etal}{et al.\,}
\newcommand{\eg}{e.g.\,}
\newcommand{\ie}{i.e.\,}

%--------------------------------------------------------------------------------
%define colour format
%--------------------------------------------------------------------------------
\newcommand{\BV}{\mbox{$(B-V)$}\,}  
\newcommand{\UG}{\mbox{$(u-g)$}\,}        
\newcommand{\GR}{\mbox{$(g-r)$}\,}
\newcommand{\GI}{\mbox{$(g-i)$}\,}
\newcommand{\RI}{\mbox{$(r-i)$}\,}
\newcommand{\RZ}{\mbox{$(r-z)$}\,}
\newcommand{\IZ}{\mbox{$(i-z)$}\,}
\newcommand{\ZJ}{\mbox{$(z-J)$}\,}
\newcommand{\VJ}{\mbox{$(V-J)$}\,}
\newcommand{\JH}{\mbox{$(J-H)$}\,}
\newcommand{\JK}{\mbox{$(J-K_S)$}\,}
\newcommand{\HK}{\mbox{$(H-K_S)$}\,}
\newcommand{\JWa}{\mbox{$(J-W1)$}\,}
\newcommand{\JWb}{\mbox{$(J-W2)$}\,}
\newcommand{\HWa}{\mbox{$(H-W1)$}\,}
\newcommand{\HWb}{\mbox{$(H-W2)$}\,}
\newcommand{\KWb}{\mbox{$(K_S-W2)$}\,}
\newcommand{\WaWb}{\mbox{$(W1-W2)$}\,}
\newcommand{\WaWc}{\mbox{$(W1-W3)$}\,}
\newcommand{\WbWc}{\mbox{$(W2-W3)$}\,}

%--------------------------------------------------------------------------------
% Misc
%--------------------------------------------------------------------------------
\newcommand{\ipython}{{\sc ipython}}
\newcommand{\python}{{\sc python}}

%--------------------------------------------------------------------------------
% Referencing
%--------------------------------------------------------------------------------
%list of abbreviations in bibfile that latex does not understand 
\input{./referencing/article_names}
%list of citing aliases
\defcitealias{abell59}{AB59}

%--------------------------------------------------------------------------------
% Acronyms, indexing and referencing sections, figures, tables, equations
%--------------------------------------------------------------------------------

% From http://tex.stackexchange.com/questions/7992/
%      command-to-uppercase-the-first-letter-of-each-word-in-a-sentence
%
% IMPORTANT NOTE index key is CASE SENSITIVE -- must have key the same in all cases
%  use \defineas{Key}{key} to enable a uppercase word or just index after with \indeX
\newcommand{\indeX}[1]{\index{#1@\protect\capitalisewords{#1}\xspace}}

%Highlight index words

%\newcommand{\define}[1]{\textcolor{Green}{#1}\indeX{#1}}
%\newcommand{\defineas}[2]{\textcolor{Green}{#1}\indeX{#2}}

% Do not highlight index words (final version)

\newcommand{\define}[1]{#1\indeX{#1}}
\newcommand{\defineas}[2]{#1\indeX{#2}}
% use in this format (must have glossary entry)
% 	\newcommand{\key}{key}{\index{key}\useglosentry{key}}
%   \newcommand{\}{}{\index{}\useglosentry{}}
\newcommand{\acro}[1]{\define{#1}\useglosentry{#1}}

%Change default name for a Figure (between Fig. and Figure)
\renewcommand{\figurename}{Figure.}
%Default report style uses bibliography not references
\renewcommand{\bibname}{References} % by default report style uses bibliography

%referencing sections, figures, tables, equations
\newcommand{\reffig}[1]{Figure \ref{#1}}
\newcommand{\reftab}[1]{Table \ref{#1}}
\newcommand{\refequ}[1]{Equation \ref{#1}}
\newcommand{\refsec}[1]{Section \ref{#1}}

%--------------------------------------------------------------------------------
% Input Tables from tex file
%--------------------------------------------------------------------------------
% define a column type that is vertically and horizontally centered
\newcolumntype{M}[1]{>{\centering\arraybackslash}m{#1}}


% Below is my way to use tables (it is completely optional)
% I store tables in separate .tex files and load them through these commands
% to save time and space I use the reference as the file name
% i.e.    ch1_table_my_first_table     would need ch1_table_my_first_table.tex

% Use \inputtable{label}{caption}

% note label must be same as table tex file name
% note tex file must be placed in ./tables/ folder

\newcommand{\inputtable}[2]{
	\begin{table}
	\begin{center}
	\input{./tables/#1}
	\caption{#2 \label{#1}}
	\end{center}
	\end{table}}%

\newcommand{\inputtableS}[3]{
	\begin{table}
	\begin{center}
	\input{./tables/#1}
	\caption[#2]{#3 \label{#1}}
	\end{center}
	\end{table}}%

\newcommand{\inputtablex}[3]{
	\begin{table}[#3]
	\begin{center}
	\input{./tables/#1}
	\caption{#2 \label{#1}}
	\end{center}
	\end{table}}%

\DeclareCaptionFormat{cont}{#1 (cont.)#2#3\par}

\newcommand{\inputtableC}[2]{
	\begin{table}
	\begin{center}
    \ContinuedFloat
    \captionsetup{list=off,format=cont}
	\input{./tables/#1}
	\caption{#2 \label{#1}}
	\end{center}
	\end{table}}%


 must be added to root tex file for these to work
%
% This is basically where I define my new commands
%
% format is normally as follows:
%
%	\newcommand{\command}{stuff here will be executed on use of command}
%
%	then in text use \command to execute "stuff"
%
% ------------------------------------------------------------------------------
% Physics
% ------------------------------------------------------------------------------
\newcommand{\arcsec}{\,\mbox{arcsec}\,}
\newcommand{\Teff}{\mbox{$T_{eff}$}\,}
\newcommand{\logg}{\mbox{$log$\,$g$}\,}
\newcommand{\Msun}{\mbox{$M_{\odot}$}\,}
\newcommand{\Mjup}{\mbox{$M_{Jup}$}\,}
\newcommand{\Rsun}{\mbox{$R_{\odot}$}\,}
\newcommand{\Lsun}{\mbox{$L_{\odot}$}\,}
\newcommand{\micron}{\mbox{${\mu}$m}\,}
%\newcommand{\arcsec}{\mbox{$^{\prime\prime}$}\,}
\newcommand{\Rv}{\mbox{$R_{\text v}$}\,}
\newcommand{\Av}{\mbox{$A_{\text v}$}\,}
\newcommand{\Alam}{\mbox{$A_{\lambda}$}\,}
\newcommand{\EBV}{\mbox{$E(B - V)$}\,}
\newcommand{\halpha}{\mbox{H$\alpha$}\,}
\newcommand{\nai}{\mbox{Na{\sc i}}\,}
\newcommand{\MJ}{\mbox{$M_{J}$}\,}
\newcommand{\MV}{\mbox{$M_{V}$}\,}
\newcommand{\HJ}{\mbox{$H_{J}$}\,}
\newcommand{\HV}{\mbox{$H_{V}$}\,}
\newcommand{\degs}{\mbox{$^{\circ}$}\,}
\newcommand{\Mpc}{\mbox{$Mpc$}\,}
\newcommand{\Myr}{\mbox{$Myr$}\,}
\newcommand{\Gyr}{\mbox{$Gyr$}\,}
%--------------------------------------------------------------------------------
% Maths
%--------------------------------------------------------------------------------
\newcommand{\simm}{$\sim$}
\newcommand{\asin}{sin$^{-1}$}
\newcommand{\la}{\lesssim}
\newcommand{\ga}{\gtrsim}
\newcommand{\chis}{$\chi$-squared}
\newcommand{\chiS}{$\chi$-Squared}
\newcommand{\tx}{\mbox{$\times10$}}

%--------------------------------------------------------------------------------
% Language
%--------------------------------------------------------------------------------
\newcommand{\etal}{et al.\,}
\newcommand{\eg}{e.g.\,}
\newcommand{\ie}{i.e.\,}

%--------------------------------------------------------------------------------
%define colour format
%--------------------------------------------------------------------------------
\newcommand{\BV}{\mbox{$(B-V)$}\,}  
\newcommand{\UG}{\mbox{$(u-g)$}\,}        
\newcommand{\GR}{\mbox{$(g-r)$}\,}
\newcommand{\GI}{\mbox{$(g-i)$}\,}
\newcommand{\RI}{\mbox{$(r-i)$}\,}
\newcommand{\RZ}{\mbox{$(r-z)$}\,}
\newcommand{\IZ}{\mbox{$(i-z)$}\,}
\newcommand{\ZJ}{\mbox{$(z-J)$}\,}
\newcommand{\VJ}{\mbox{$(V-J)$}\,}
\newcommand{\JH}{\mbox{$(J-H)$}\,}
\newcommand{\JK}{\mbox{$(J-K_S)$}\,}
\newcommand{\HK}{\mbox{$(H-K_S)$}\,}
\newcommand{\JWa}{\mbox{$(J-W1)$}\,}
\newcommand{\JWb}{\mbox{$(J-W2)$}\,}
\newcommand{\HWa}{\mbox{$(H-W1)$}\,}
\newcommand{\HWb}{\mbox{$(H-W2)$}\,}
\newcommand{\KWb}{\mbox{$(K_S-W2)$}\,}
\newcommand{\WaWb}{\mbox{$(W1-W2)$}\,}
\newcommand{\WaWc}{\mbox{$(W1-W3)$}\,}
\newcommand{\WbWc}{\mbox{$(W2-W3)$}\,}

%--------------------------------------------------------------------------------
% Misc
%--------------------------------------------------------------------------------
\newcommand{\ipython}{{\sc ipython}}
\newcommand{\python}{{\sc python}}

%--------------------------------------------------------------------------------
% Referencing
%--------------------------------------------------------------------------------
%list of abbreviations in bibfile that latex does not understand 
% Bibliography and bibfile
\def\aj{AJ}%
          % Astronomical Journal
\def\actaa{Acta Astron.}%
          % Acta Astronomica
\def\araa{ARA\&A}%
          % Annual Review of Astron and Astrophys
\def\apj{ApJ}%
          % Astrophysical Journal
\def\apjl{ApJ}%
          % Astrophysical Journal, Letters
\def\apjs{ApJS}%
          % Astrophysical Journal, Supplement
\def\ao{Appl.~Opt.}%
          % Applied Optics
\def\apss{Ap\&SS}%
          % Astrophysics and Space Science
\def\aap{A\&A}%
          % Astronomy and Astrophysics
\def\aapr{A\&A~Rev.}%
          % Astronomy and Astrophysics Reviews
\def\aaps{A\&AS}%
          % Astronomy and Astrophysics, Supplement
\def\azh{AZh}%
          % Astronomicheskii Zhurnal
\def\baas{BAAS}%
          % Bulletin of the AAS
\def\bac{Bull. astr. Inst. Czechosl.}%
          % Bulletin of the Astronomical Institutes of Czechoslovakia 
\def\caa{Chinese Astron. Astrophys.}%
          % Chinese Astronomy and Astrophysics
\def\cjaa{Chinese J. Astron. Astrophys.}%
          % Chinese Journal of Astronomy and Astrophysics
\def\icarus{Icarus}%
          % Icarus
\def\jcap{J. Cosmology Astropart. Phys.}%
          % Journal of Cosmology and Astroparticle Physics
\def\jrasc{JRASC}%
          % Journal of the RAS of Canada
\def\mnras{MNRAS}%
          % Monthly Notices of the RAS
\def\memras{MmRAS}%
          % Memoirs of the RAS
\def\na{New A}%
          % New Astronomy
\def\nar{New A Rev.}%
          % New Astronomy Review
\def\pasa{PASA}%
          % Publications of the Astron. Soc. of Australia
\def\pra{Phys.~Rev.~A}%
          % Physical Review A: General Physics
\def\prb{Phys.~Rev.~B}%
          % Physical Review B: Solid State
\def\prc{Phys.~Rev.~C}%
          % Physical Review C
\def\prd{Phys.~Rev.~D}%
          % Physical Review D
\def\pre{Phys.~Rev.~E}%
          % Physical Review E
\def\prl{Phys.~Rev.~Lett.}%
          % Physical Review Letters
\def\pasp{PASP}%
          % Publications of the ASP
\def\pasj{PASJ}%
          % Publications of the ASJ
\def\qjras{QJRAS}%
          % Quarterly Journal of the RAS
\def\rmxaa{Rev. Mexicana Astron. Astrofis.}%
          % Revista Mexicana de Astronomia y Astrofisica
\def\skytel{S\&T}%
          % Sky and Telescope
\def\solphys{Sol.~Phys.}%
          % Solar Physics
\def\sovast{Soviet~Ast.}%
          % Soviet Astronomy
\def\ssr{Space~Sci.~Rev.}%
          % Space Science Reviews
\def\zap{ZAp}%
          % Zeitschrift fuer Astrophysik
\def\nat{Nature}%
          % Nature
\def\iaucirc{IAU~Circ.}%
          % IAU Cirulars
\def\aplett{Astrophys.~Lett.}%
          % Astrophysics Letters
\def\apspr{Astrophys.~Space~Phys.~Res.}%
          % Astrophysics Space Physics Research
\def\bain{Bull.~Astron.~Inst.~Netherlands}%
          % Bulletin Astronomical Institute of the Netherlands
\def\fcp{Fund.~Cosmic~Phys.}%
          % Fundamental Cosmic Physics
\def\gca{Geochim.~Cosmochim.~Acta}%
          % Geochimica Cosmochimica Acta
\def\grl{Geophys.~Res.~Lett.}%
          % Geophysics Research Letters
\def\jcp{J.~Chem.~Phys.}%
          % Journal of Chemical Physics
\def\jgr{J.~Geophys.~Res.}%
          % Journal of Geophysics Research
\def\jqsrt{J.~Quant.~Spec.~Radiat.~Transf.}%
          % Journal of Quantitiative Spectroscopy and Radiative Trasfer
\def\memsai{Mem.~Soc.~Astron.~Italiana}%
          % Mem. Societa Astronomica Italiana
\def\nphysa{Nucl.~Phys.~A}%
          % Nuclear Physics A
\def\physrep{Phys.~Rep.}%
          % Physics Reports
\def\physscr{Phys.~Scr}%
          % Physica Scripta
\def\planss{Planet.~Space~Sci.}%
          % Planetary Space Science
\def\procspie{Proc.~SPIE}%
          % Proceedings of the SPIE
\let\astap=\aap
\let\apjlett=\apjl
\let\apjsupp=\apjs
\let\applopt=\ao

%list of citing aliases
\defcitealias{abell59}{AB59}

%--------------------------------------------------------------------------------
% Acronyms, indexing and referencing sections, figures, tables, equations
%--------------------------------------------------------------------------------

% From http://tex.stackexchange.com/questions/7992/
%      command-to-uppercase-the-first-letter-of-each-word-in-a-sentence
%
% IMPORTANT NOTE index key is CASE SENSITIVE -- must have key the same in all cases
%  use \defineas{Key}{key} to enable a uppercase word or just index after with \indeX
\newcommand{\indeX}[1]{\index{#1@\protect\capitalisewords{#1}\xspace}}

%Highlight index words

%\newcommand{\define}[1]{\textcolor{Green}{#1}\indeX{#1}}
%\newcommand{\defineas}[2]{\textcolor{Green}{#1}\indeX{#2}}

% Do not highlight index words (final version)

\newcommand{\define}[1]{#1\indeX{#1}}
\newcommand{\defineas}[2]{#1\indeX{#2}}
% use in this format (must have glossary entry)
% 	\newcommand{\key}{key}{\index{key}\useglosentry{key}}
%   \newcommand{\}{}{\index{}\useglosentry{}}
\newcommand{\acro}[1]{\define{#1}\useglosentry{#1}}

%Change default name for a Figure (between Fig. and Figure)
\renewcommand{\figurename}{Figure.}
%Default report style uses bibliography not references
\renewcommand{\bibname}{References} % by default report style uses bibliography

%referencing sections, figures, tables, equations
\newcommand{\reffig}[1]{Figure \ref{#1}}
\newcommand{\reftab}[1]{Table \ref{#1}}
\newcommand{\refequ}[1]{Equation \ref{#1}}
\newcommand{\refsec}[1]{Section \ref{#1}}

%--------------------------------------------------------------------------------
% Input Tables from tex file
%--------------------------------------------------------------------------------
% define a column type that is vertically and horizontally centered
\newcolumntype{M}[1]{>{\centering\arraybackslash}m{#1}}


% Below is my way to use tables (it is completely optional)
% I store tables in separate .tex files and load them through these commands
% to save time and space I use the reference as the file name
% i.e.    ch1_table_my_first_table     would need ch1_table_my_first_table.tex

% Use \inputtable{label}{caption}

% note label must be same as table tex file name
% note tex file must be placed in ./tables/ folder

\newcommand{\inputtable}[2]{
	\begin{table}
	\begin{center}
	\input{./tables/#1}
	\caption{#2 \label{#1}}
	\end{center}
	\end{table}}%

\newcommand{\inputtableS}[3]{
	\begin{table}
	\begin{center}
	\input{./tables/#1}
	\caption[#2]{#3 \label{#1}}
	\end{center}
	\end{table}}%

\newcommand{\inputtablex}[3]{
	\begin{table}[#3]
	\begin{center}
	\input{./tables/#1}
	\caption{#2 \label{#1}}
	\end{center}
	\end{table}}%

\DeclareCaptionFormat{cont}{#1 (cont.)#2#3\par}

\newcommand{\inputtableC}[2]{
	\begin{table}
	\begin{center}
    \ContinuedFloat
    \captionsetup{list=off,format=cont}
	\input{./tables/#1}
	\caption{#2 \label{#1}}
	\end{center}
	\end{table}}%


 must be added to root tex file for these to work
%
% This is basically where I define my new commands
%
% format is normally as follows:
%
%	\newcommand{\command}{stuff here will be executed on use of command}
%
%	then in text use \command to execute "stuff"
%
% ------------------------------------------------------------------------------
% Physics
% ------------------------------------------------------------------------------
\newcommand{\arcsec}{\,\mbox{arcsec}\,}
\newcommand{\Teff}{\mbox{$T_{eff}$}\,}
\newcommand{\logg}{\mbox{$log$\,$g$}\,}
\newcommand{\Msun}{\mbox{$M_{\odot}$}\,}
\newcommand{\Mjup}{\mbox{$M_{Jup}$}\,}
\newcommand{\Rsun}{\mbox{$R_{\odot}$}\,}
\newcommand{\Lsun}{\mbox{$L_{\odot}$}\,}
\newcommand{\micron}{\mbox{${\mu}$m}\,}
%\newcommand{\arcsec}{\mbox{$^{\prime\prime}$}\,}
\newcommand{\Rv}{\mbox{$R_{\text v}$}\,}
\newcommand{\Av}{\mbox{$A_{\text v}$}\,}
\newcommand{\Alam}{\mbox{$A_{\lambda}$}\,}
\newcommand{\EBV}{\mbox{$E(B - V)$}\,}
\newcommand{\halpha}{\mbox{H$\alpha$}\,}
\newcommand{\nai}{\mbox{Na{\sc i}}\,}
\newcommand{\MJ}{\mbox{$M_{J}$}\,}
\newcommand{\MV}{\mbox{$M_{V}$}\,}
\newcommand{\HJ}{\mbox{$H_{J}$}\,}
\newcommand{\HV}{\mbox{$H_{V}$}\,}
\newcommand{\degs}{\mbox{$^{\circ}$}\,}
\newcommand{\Mpc}{\mbox{$Mpc$}\,}
\newcommand{\Myr}{\mbox{$Myr$}\,}
\newcommand{\Gyr}{\mbox{$Gyr$}\,}
%--------------------------------------------------------------------------------
% Maths
%--------------------------------------------------------------------------------
\newcommand{\simm}{$\sim$}
\newcommand{\asin}{sin$^{-1}$}
\newcommand{\la}{\lesssim}
\newcommand{\ga}{\gtrsim}
\newcommand{\chis}{$\chi$-squared}
\newcommand{\chiS}{$\chi$-Squared}
\newcommand{\tx}{\mbox{$\times10$}}

%--------------------------------------------------------------------------------
% Language
%--------------------------------------------------------------------------------
\newcommand{\etal}{et al.\,}
\newcommand{\eg}{e.g.\,}
\newcommand{\ie}{i.e.\,}

%--------------------------------------------------------------------------------
%define colour format
%--------------------------------------------------------------------------------
\newcommand{\BV}{\mbox{$(B-V)$}\,}  
\newcommand{\UG}{\mbox{$(u-g)$}\,}        
\newcommand{\GR}{\mbox{$(g-r)$}\,}
\newcommand{\GI}{\mbox{$(g-i)$}\,}
\newcommand{\RI}{\mbox{$(r-i)$}\,}
\newcommand{\RZ}{\mbox{$(r-z)$}\,}
\newcommand{\IZ}{\mbox{$(i-z)$}\,}
\newcommand{\ZJ}{\mbox{$(z-J)$}\,}
\newcommand{\VJ}{\mbox{$(V-J)$}\,}
\newcommand{\JH}{\mbox{$(J-H)$}\,}
\newcommand{\JK}{\mbox{$(J-K_S)$}\,}
\newcommand{\HK}{\mbox{$(H-K_S)$}\,}
\newcommand{\JWa}{\mbox{$(J-W1)$}\,}
\newcommand{\JWb}{\mbox{$(J-W2)$}\,}
\newcommand{\HWa}{\mbox{$(H-W1)$}\,}
\newcommand{\HWb}{\mbox{$(H-W2)$}\,}
\newcommand{\KWb}{\mbox{$(K_S-W2)$}\,}
\newcommand{\WaWb}{\mbox{$(W1-W2)$}\,}
\newcommand{\WaWc}{\mbox{$(W1-W3)$}\,}
\newcommand{\WbWc}{\mbox{$(W2-W3)$}\,}

%--------------------------------------------------------------------------------
% Misc
%--------------------------------------------------------------------------------
\newcommand{\ipython}{{\sc ipython}}
\newcommand{\python}{{\sc python}}

%--------------------------------------------------------------------------------
% Referencing
%--------------------------------------------------------------------------------
%list of abbreviations in bibfile that latex does not understand 
% Bibliography and bibfile
\def\aj{AJ}%
          % Astronomical Journal
\def\actaa{Acta Astron.}%
          % Acta Astronomica
\def\araa{ARA\&A}%
          % Annual Review of Astron and Astrophys
\def\apj{ApJ}%
          % Astrophysical Journal
\def\apjl{ApJ}%
          % Astrophysical Journal, Letters
\def\apjs{ApJS}%
          % Astrophysical Journal, Supplement
\def\ao{Appl.~Opt.}%
          % Applied Optics
\def\apss{Ap\&SS}%
          % Astrophysics and Space Science
\def\aap{A\&A}%
          % Astronomy and Astrophysics
\def\aapr{A\&A~Rev.}%
          % Astronomy and Astrophysics Reviews
\def\aaps{A\&AS}%
          % Astronomy and Astrophysics, Supplement
\def\azh{AZh}%
          % Astronomicheskii Zhurnal
\def\baas{BAAS}%
          % Bulletin of the AAS
\def\bac{Bull. astr. Inst. Czechosl.}%
          % Bulletin of the Astronomical Institutes of Czechoslovakia 
\def\caa{Chinese Astron. Astrophys.}%
          % Chinese Astronomy and Astrophysics
\def\cjaa{Chinese J. Astron. Astrophys.}%
          % Chinese Journal of Astronomy and Astrophysics
\def\icarus{Icarus}%
          % Icarus
\def\jcap{J. Cosmology Astropart. Phys.}%
          % Journal of Cosmology and Astroparticle Physics
\def\jrasc{JRASC}%
          % Journal of the RAS of Canada
\def\mnras{MNRAS}%
          % Monthly Notices of the RAS
\def\memras{MmRAS}%
          % Memoirs of the RAS
\def\na{New A}%
          % New Astronomy
\def\nar{New A Rev.}%
          % New Astronomy Review
\def\pasa{PASA}%
          % Publications of the Astron. Soc. of Australia
\def\pra{Phys.~Rev.~A}%
          % Physical Review A: General Physics
\def\prb{Phys.~Rev.~B}%
          % Physical Review B: Solid State
\def\prc{Phys.~Rev.~C}%
          % Physical Review C
\def\prd{Phys.~Rev.~D}%
          % Physical Review D
\def\pre{Phys.~Rev.~E}%
          % Physical Review E
\def\prl{Phys.~Rev.~Lett.}%
          % Physical Review Letters
\def\pasp{PASP}%
          % Publications of the ASP
\def\pasj{PASJ}%
          % Publications of the ASJ
\def\qjras{QJRAS}%
          % Quarterly Journal of the RAS
\def\rmxaa{Rev. Mexicana Astron. Astrofis.}%
          % Revista Mexicana de Astronomia y Astrofisica
\def\skytel{S\&T}%
          % Sky and Telescope
\def\solphys{Sol.~Phys.}%
          % Solar Physics
\def\sovast{Soviet~Ast.}%
          % Soviet Astronomy
\def\ssr{Space~Sci.~Rev.}%
          % Space Science Reviews
\def\zap{ZAp}%
          % Zeitschrift fuer Astrophysik
\def\nat{Nature}%
          % Nature
\def\iaucirc{IAU~Circ.}%
          % IAU Cirulars
\def\aplett{Astrophys.~Lett.}%
          % Astrophysics Letters
\def\apspr{Astrophys.~Space~Phys.~Res.}%
          % Astrophysics Space Physics Research
\def\bain{Bull.~Astron.~Inst.~Netherlands}%
          % Bulletin Astronomical Institute of the Netherlands
\def\fcp{Fund.~Cosmic~Phys.}%
          % Fundamental Cosmic Physics
\def\gca{Geochim.~Cosmochim.~Acta}%
          % Geochimica Cosmochimica Acta
\def\grl{Geophys.~Res.~Lett.}%
          % Geophysics Research Letters
\def\jcp{J.~Chem.~Phys.}%
          % Journal of Chemical Physics
\def\jgr{J.~Geophys.~Res.}%
          % Journal of Geophysics Research
\def\jqsrt{J.~Quant.~Spec.~Radiat.~Transf.}%
          % Journal of Quantitiative Spectroscopy and Radiative Trasfer
\def\memsai{Mem.~Soc.~Astron.~Italiana}%
          % Mem. Societa Astronomica Italiana
\def\nphysa{Nucl.~Phys.~A}%
          % Nuclear Physics A
\def\physrep{Phys.~Rep.}%
          % Physics Reports
\def\physscr{Phys.~Scr}%
          % Physica Scripta
\def\planss{Planet.~Space~Sci.}%
          % Planetary Space Science
\def\procspie{Proc.~SPIE}%
          % Proceedings of the SPIE
\let\astap=\aap
\let\apjlett=\apjl
\let\apjsupp=\apjs
\let\applopt=\ao

%list of citing aliases
\defcitealias{abell59}{AB59}

%--------------------------------------------------------------------------------
% Acronyms, indexing and referencing sections, figures, tables, equations
%--------------------------------------------------------------------------------

% From http://tex.stackexchange.com/questions/7992/
%      command-to-uppercase-the-first-letter-of-each-word-in-a-sentence
%
% IMPORTANT NOTE index key is CASE SENSITIVE -- must have key the same in all cases
%  use \defineas{Key}{key} to enable a uppercase word or just index after with \indeX
\newcommand{\indeX}[1]{\index{#1@\protect\capitalisewords{#1}\xspace}}

%Highlight index words

%\newcommand{\define}[1]{\textcolor{Green}{#1}\indeX{#1}}
%\newcommand{\defineas}[2]{\textcolor{Green}{#1}\indeX{#2}}

% Do not highlight index words (final version)

\newcommand{\define}[1]{#1\indeX{#1}}
\newcommand{\defineas}[2]{#1\indeX{#2}}
% use in this format (must have glossary entry)
% 	\newcommand{\key}{key}{\index{key}\useglosentry{key}}
%   \newcommand{\}{}{\index{}\useglosentry{}}
\newcommand{\acro}[1]{\define{#1}\useglosentry{#1}}

%Change default name for a Figure (between Fig. and Figure)
\renewcommand{\figurename}{Figure.}
%Default report style uses bibliography not references
\renewcommand{\bibname}{References} % by default report style uses bibliography

%referencing sections, figures, tables, equations
\newcommand{\reffig}[1]{Figure \ref{#1}}
\newcommand{\reftab}[1]{Table \ref{#1}}
\newcommand{\refequ}[1]{Equation \ref{#1}}
\newcommand{\refsec}[1]{Section \ref{#1}}

%--------------------------------------------------------------------------------
% Input Tables from tex file
%--------------------------------------------------------------------------------
% define a column type that is vertically and horizontally centered
\newcolumntype{M}[1]{>{\centering\arraybackslash}m{#1}}


% Below is my way to use tables (it is completely optional)
% I store tables in separate .tex files and load them through these commands
% to save time and space I use the reference as the file name
% i.e.    ch1_table_my_first_table     would need ch1_table_my_first_table.tex

% Use \inputtable{label}{caption}

% note label must be same as table tex file name
% note tex file must be placed in ./tables/ folder

\newcommand{\inputtable}[2]{
	\begin{table}
	\begin{center}
	\input{./tables/#1}
	\caption{#2 \label{#1}}
	\end{center}
	\end{table}}%

\newcommand{\inputtableS}[3]{
	\begin{table}
	\begin{center}
	\input{./tables/#1}
	\caption[#2]{#3 \label{#1}}
	\end{center}
	\end{table}}%

\newcommand{\inputtablex}[3]{
	\begin{table}[#3]
	\begin{center}
	\input{./tables/#1}
	\caption{#2 \label{#1}}
	\end{center}
	\end{table}}%

\DeclareCaptionFormat{cont}{#1 (cont.)#2#3\par}

\newcommand{\inputtableC}[2]{
	\begin{table}
	\begin{center}
    \ContinuedFloat
    \captionsetup{list=off,format=cont}
	\input{./tables/#1}
	\caption{#2 \label{#1}}
	\end{center}
	\end{table}}%


 must be added to root tex file for these to work
%
% This is basically where I define my new commands
%
% format is normally as follows:
%
%	\newcommand{\command}{stuff here will be executed on use of command}
%
%	then in text use \command to execute "stuff"
%
% ------------------------------------------------------------------------------
% Physics
% ------------------------------------------------------------------------------
\newcommand{\arcsec}{\,\mbox{arcsec}\,}
\newcommand{\Teff}{\mbox{$T_{eff}$}\,}
\newcommand{\logg}{\mbox{$log$\,$g$}\,}
\newcommand{\Msun}{\mbox{$M_{\odot}$}\,}
\newcommand{\Mjup}{\mbox{$M_{Jup}$}\,}
\newcommand{\Rsun}{\mbox{$R_{\odot}$}\,}
\newcommand{\Lsun}{\mbox{$L_{\odot}$}\,}
\newcommand{\micron}{\mbox{${\mu}$m}\,}
%\newcommand{\arcsec}{\mbox{$^{\prime\prime}$}\,}
\newcommand{\Rv}{\mbox{$R_{\text v}$}\,}
\newcommand{\Av}{\mbox{$A_{\text v}$}\,}
\newcommand{\Alam}{\mbox{$A_{\lambda}$}\,}
\newcommand{\EBV}{\mbox{$E(B - V)$}\,}
\newcommand{\halpha}{\mbox{H$\alpha$}\,}
\newcommand{\nai}{\mbox{Na{\sc i}}\,}
\newcommand{\MJ}{\mbox{$M_{J}$}\,}
\newcommand{\MV}{\mbox{$M_{V}$}\,}
\newcommand{\HJ}{\mbox{$H_{J}$}\,}
\newcommand{\HV}{\mbox{$H_{V}$}\,}
\newcommand{\degs}{\mbox{$^{\circ}$}\,}
\newcommand{\Mpc}{\mbox{$Mpc$}\,}
\newcommand{\Myr}{\mbox{$Myr$}\,}
\newcommand{\Gyr}{\mbox{$Gyr$}\,}
%--------------------------------------------------------------------------------
% Maths
%--------------------------------------------------------------------------------
\newcommand{\simm}{$\sim$}
\newcommand{\asin}{sin$^{-1}$}
\newcommand{\la}{\lesssim}
\newcommand{\ga}{\gtrsim}
\newcommand{\chis}{$\chi$-squared}
\newcommand{\chiS}{$\chi$-Squared}
\newcommand{\tx}{\mbox{$\times10$}}

%--------------------------------------------------------------------------------
% Language
%--------------------------------------------------------------------------------
\newcommand{\etal}{et al.\,}
\newcommand{\eg}{e.g.\,}
\newcommand{\ie}{i.e.\,}

%--------------------------------------------------------------------------------
%define colour format
%--------------------------------------------------------------------------------
\newcommand{\BV}{\mbox{$(B-V)$}\,}  
\newcommand{\UG}{\mbox{$(u-g)$}\,}        
\newcommand{\GR}{\mbox{$(g-r)$}\,}
\newcommand{\GI}{\mbox{$(g-i)$}\,}
\newcommand{\RI}{\mbox{$(r-i)$}\,}
\newcommand{\RZ}{\mbox{$(r-z)$}\,}
\newcommand{\IZ}{\mbox{$(i-z)$}\,}
\newcommand{\ZJ}{\mbox{$(z-J)$}\,}
\newcommand{\VJ}{\mbox{$(V-J)$}\,}
\newcommand{\JH}{\mbox{$(J-H)$}\,}
\newcommand{\JK}{\mbox{$(J-K_S)$}\,}
\newcommand{\HK}{\mbox{$(H-K_S)$}\,}
\newcommand{\JWa}{\mbox{$(J-W1)$}\,}
\newcommand{\JWb}{\mbox{$(J-W2)$}\,}
\newcommand{\HWa}{\mbox{$(H-W1)$}\,}
\newcommand{\HWb}{\mbox{$(H-W2)$}\,}
\newcommand{\KWb}{\mbox{$(K_S-W2)$}\,}
\newcommand{\WaWb}{\mbox{$(W1-W2)$}\,}
\newcommand{\WaWc}{\mbox{$(W1-W3)$}\,}
\newcommand{\WbWc}{\mbox{$(W2-W3)$}\,}

%--------------------------------------------------------------------------------
% Misc
%--------------------------------------------------------------------------------
\newcommand{\ipython}{{\sc ipython}}
\newcommand{\python}{{\sc python}}

%--------------------------------------------------------------------------------
% Referencing
%--------------------------------------------------------------------------------
%list of abbreviations in bibfile that latex does not understand 
% Bibliography and bibfile
\def\aj{AJ}%
          % Astronomical Journal
\def\actaa{Acta Astron.}%
          % Acta Astronomica
\def\araa{ARA\&A}%
          % Annual Review of Astron and Astrophys
\def\apj{ApJ}%
          % Astrophysical Journal
\def\apjl{ApJ}%
          % Astrophysical Journal, Letters
\def\apjs{ApJS}%
          % Astrophysical Journal, Supplement
\def\ao{Appl.~Opt.}%
          % Applied Optics
\def\apss{Ap\&SS}%
          % Astrophysics and Space Science
\def\aap{A\&A}%
          % Astronomy and Astrophysics
\def\aapr{A\&A~Rev.}%
          % Astronomy and Astrophysics Reviews
\def\aaps{A\&AS}%
          % Astronomy and Astrophysics, Supplement
\def\azh{AZh}%
          % Astronomicheskii Zhurnal
\def\baas{BAAS}%
          % Bulletin of the AAS
\def\bac{Bull. astr. Inst. Czechosl.}%
          % Bulletin of the Astronomical Institutes of Czechoslovakia 
\def\caa{Chinese Astron. Astrophys.}%
          % Chinese Astronomy and Astrophysics
\def\cjaa{Chinese J. Astron. Astrophys.}%
          % Chinese Journal of Astronomy and Astrophysics
\def\icarus{Icarus}%
          % Icarus
\def\jcap{J. Cosmology Astropart. Phys.}%
          % Journal of Cosmology and Astroparticle Physics
\def\jrasc{JRASC}%
          % Journal of the RAS of Canada
\def\mnras{MNRAS}%
          % Monthly Notices of the RAS
\def\memras{MmRAS}%
          % Memoirs of the RAS
\def\na{New A}%
          % New Astronomy
\def\nar{New A Rev.}%
          % New Astronomy Review
\def\pasa{PASA}%
          % Publications of the Astron. Soc. of Australia
\def\pra{Phys.~Rev.~A}%
          % Physical Review A: General Physics
\def\prb{Phys.~Rev.~B}%
          % Physical Review B: Solid State
\def\prc{Phys.~Rev.~C}%
          % Physical Review C
\def\prd{Phys.~Rev.~D}%
          % Physical Review D
\def\pre{Phys.~Rev.~E}%
          % Physical Review E
\def\prl{Phys.~Rev.~Lett.}%
          % Physical Review Letters
\def\pasp{PASP}%
          % Publications of the ASP
\def\pasj{PASJ}%
          % Publications of the ASJ
\def\qjras{QJRAS}%
          % Quarterly Journal of the RAS
\def\rmxaa{Rev. Mexicana Astron. Astrofis.}%
          % Revista Mexicana de Astronomia y Astrofisica
\def\skytel{S\&T}%
          % Sky and Telescope
\def\solphys{Sol.~Phys.}%
          % Solar Physics
\def\sovast{Soviet~Ast.}%
          % Soviet Astronomy
\def\ssr{Space~Sci.~Rev.}%
          % Space Science Reviews
\def\zap{ZAp}%
          % Zeitschrift fuer Astrophysik
\def\nat{Nature}%
          % Nature
\def\iaucirc{IAU~Circ.}%
          % IAU Cirulars
\def\aplett{Astrophys.~Lett.}%
          % Astrophysics Letters
\def\apspr{Astrophys.~Space~Phys.~Res.}%
          % Astrophysics Space Physics Research
\def\bain{Bull.~Astron.~Inst.~Netherlands}%
          % Bulletin Astronomical Institute of the Netherlands
\def\fcp{Fund.~Cosmic~Phys.}%
          % Fundamental Cosmic Physics
\def\gca{Geochim.~Cosmochim.~Acta}%
          % Geochimica Cosmochimica Acta
\def\grl{Geophys.~Res.~Lett.}%
          % Geophysics Research Letters
\def\jcp{J.~Chem.~Phys.}%
          % Journal of Chemical Physics
\def\jgr{J.~Geophys.~Res.}%
          % Journal of Geophysics Research
\def\jqsrt{J.~Quant.~Spec.~Radiat.~Transf.}%
          % Journal of Quantitiative Spectroscopy and Radiative Trasfer
\def\memsai{Mem.~Soc.~Astron.~Italiana}%
          % Mem. Societa Astronomica Italiana
\def\nphysa{Nucl.~Phys.~A}%
          % Nuclear Physics A
\def\physrep{Phys.~Rep.}%
          % Physics Reports
\def\physscr{Phys.~Scr}%
          % Physica Scripta
\def\planss{Planet.~Space~Sci.}%
          % Planetary Space Science
\def\procspie{Proc.~SPIE}%
          % Proceedings of the SPIE
\let\astap=\aap
\let\apjlett=\apjl
\let\apjsupp=\apjs
\let\applopt=\ao

%list of citing aliases
\defcitealias{abell59}{AB59}

%--------------------------------------------------------------------------------
% Acronyms, indexing and referencing sections, figures, tables, equations
%--------------------------------------------------------------------------------

% From http://tex.stackexchange.com/questions/7992/
%      command-to-uppercase-the-first-letter-of-each-word-in-a-sentence
%
% IMPORTANT NOTE index key is CASE SENSITIVE -- must have key the same in all cases
%  use \defineas{Key}{key} to enable a uppercase word or just index after with \indeX
\newcommand{\indeX}[1]{\index{#1@\protect\capitalisewords{#1}\xspace}}

%Highlight index words

%\newcommand{\define}[1]{\textcolor{Green}{#1}\indeX{#1}}
%\newcommand{\defineas}[2]{\textcolor{Green}{#1}\indeX{#2}}

% Do not highlight index words (final version)

\newcommand{\define}[1]{#1\indeX{#1}}
\newcommand{\defineas}[2]{#1\indeX{#2}}
% use in this format (must have glossary entry)
% 	\newcommand{\key}{key}{\index{key}\useglosentry{key}}
%   \newcommand{\}{}{\index{}\useglosentry{}}
\newcommand{\acro}[1]{\define{#1}\useglosentry{#1}}

%Change default name for a Figure (between Fig. and Figure)
\renewcommand{\figurename}{Figure.}
%Default report style uses bibliography not references
\renewcommand{\bibname}{References} % by default report style uses bibliography

%referencing sections, figures, tables, equations
\newcommand{\reffig}[1]{Figure \ref{#1}}
\newcommand{\reftab}[1]{Table \ref{#1}}
\newcommand{\refequ}[1]{Equation \ref{#1}}
\newcommand{\refsec}[1]{Section \ref{#1}}

%--------------------------------------------------------------------------------
% Input Tables from tex file
%--------------------------------------------------------------------------------
% define a column type that is vertically and horizontally centered
\newcolumntype{M}[1]{>{\centering\arraybackslash}m{#1}}


% Below is my way to use tables (it is completely optional)
% I store tables in separate .tex files and load them through these commands
% to save time and space I use the reference as the file name
% i.e.    ch1_table_my_first_table     would need ch1_table_my_first_table.tex

% Use \inputtable{label}{caption}

% note label must be same as table tex file name
% note tex file must be placed in ./tables/ folder

\newcommand{\inputtable}[2]{
	\begin{table}
	\begin{center}
	\input{./tables/#1}
	\caption{#2 \label{#1}}
	\end{center}
	\end{table}}%

\newcommand{\inputtableS}[3]{
	\begin{table}
	\begin{center}
	\input{./tables/#1}
	\caption[#2]{#3 \label{#1}}
	\end{center}
	\end{table}}%

\newcommand{\inputtablex}[3]{
	\begin{table}[#3]
	\begin{center}
	\input{./tables/#1}
	\caption{#2 \label{#1}}
	\end{center}
	\end{table}}%

\DeclareCaptionFormat{cont}{#1 (cont.)#2#3\par}

\newcommand{\inputtableC}[2]{
	\begin{table}
	\begin{center}
    \ContinuedFloat
    \captionsetup{list=off,format=cont}
	\input{./tables/#1}
	\caption{#2 \label{#1}}
	\end{center}
	\end{table}}%


